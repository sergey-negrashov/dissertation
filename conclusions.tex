\section{Application of Napali in Other domains}\label{sec:application-of-napali-in-other-domains}
Sensor networks are prolific in today's world.
Industrial process and environment monitoring is striving to make the world more efficient and productive.
Medical and Personal sensing is a welcome addition to in improving health care and quality of life.
Many of these fields operate in regimes suitable for Napali.
To reiterate Napali is well suited for sensor networks operating in domains with:
\begin{itemize}
    \item Signal to noise ratio of $>1$.
    \item Consensus based event detection.
    \item Two way communication between the device and sink is possible.
\end{itemize}
Any monitoring situation which requires a consensus of multiple devices, with individual devices unable to ascertain the validity of an anomaly is well suited for Napali deployment.
In this section we review potential applications of Napali to several domains intrinsically different from power quality monitoring, while still following the above requirements.

\subsection{Earthquake detection}
Detection of seismic phenomenon is a tak well suited for sensor networks.
Single location monitoring is unfeasible, due to multitude of factors.
The local noise from human activity results in a large number of false positives.
Furthermore, single location monitoring is useless for development of an early warning system.
Earthquake detection relies on prompt detection of P-waves, or pressure waves.
These waves travel faster then their more destructive counterpart: S-waves.
A prompt detection and characterisation of a P-wave can provide an early warning of an impending catastrophe.
A large number of sensor networks of varying complexity and sophistication have been deployed in order to monitor their ares of effect for seismic activity.\cite{burkett2014shakealert}\cite{zaicenco2012lessons}\cite{klapez2018first}\cite{finazzi2017statistical}

In their paper "Lessons Learned from Operating an On-site Earthquake Early Warning System"\cite{zaicenco2012lessons} authors Zaicenco and Weir-Jones describe main challenges for designing and operating an earthquake detection system:
\begin{enumerate}
    \item Unknown direction of a potential seismic event, since sources that are capable of generating
    the ground motion that exceeds design parameters are spread around the region.
    \item Multiple sources of industrial noise at the site: highway traffic, railroad, fishery, heavy trucks
    driving several meters away from the instrumented area;
    \item Requirement for the system to operate 24/7 in the autonomous mode for several years;
    \item High cost of a potential false alarm, which might result in closing the traffic on the major
    highway.
\end{enumerate}

Their experience comes from operation of borehole sensor arrays located along the highways of British Columbia.
Each sensor array is connected via a fiber drop to the central computer.
Every measurement performed by the sensor array is transmitted to main processing unit for P-wave analysis.
The triggering algorithm first precomputes a p-wave metric for each sensor, and uses a threshold based algorithm from earthquake detection.
This is a well established system which was able to detect and provide early warning for multiple earthquakes during its operational phase 2009-2011.
The design of this system is very similar to a Naive triggering method for Power Quality monitoring.
All data is funneled to the central sink and processed on site.

California Integrated Seismic Network is another seismograph sensor network consisting of over 400 high quality borehole sensors.\cite{uhrhammer2011california}
Similarly to the On-site Earthquake Early Warning System, California Integrated Seismic Network transmits all of the sensor data to the central sink.
California Integrated Seismic Network is now a part of ShakeAlert, a US based effort to integrate seismic prediction into actionable intelligence.

Another approach to earthquake monitoring comes from the newly emerged IOT domain.
Cellphone based Earthquake Network \cite{finazzi2017statistical} utilizes smartphone accelerometers in order to detect P-wave propagation throughout the world.
As a part of the Earthquake Network, cellphones which are at rest and plugged into a power source will monitor the internal accelerometer for abnormalities.
If the inertial tensor recorded by the cellphone passes a threshold a message will be transmitted to a central cloudbased sink.
The cloudbased sink in turn uses device location and statistical clustering in order to determine if a P-wave has been detected.
Another IOT sensor network designed for earthquake detection is called Earthcloud.\cite{klapez2018first}
This sensor network utilises dedicated low cost sensors which communicate via the Internet to the centralized sink.
Similarly to the Earthquake Network, Earthcloud utilizes the number of "prewarnings"(devices which passed the local threhold) in order to determine if an earthquake is taking place.

Sensor networks described above fall into the two categories described in the previous chapters.
The On-site Earthquake Early Warning System and California Integrated Seismic Network are the Naive approaches with all of the sensor data funneled to the sink.
Earthcloud and Earthquake Network on the other hand are similar to the self-triggered system, with additional statistical analysis performed at the sink.
All three networks maintain two way TCP/IP link between devices and the sink.
With a relatively number of nodes On-site Earthquake Early Warning System and California Integrated Seismic Network are able to operate in the Naive mode due to relatively low bandwidth requirements.
On the other hand, Earthquake Network app has 4 million downloads and 75000 active cellphones, and as such is forced to operate in the self triggered mode.

Napali methodology could further enchance both of these event detection topologies.
"High quality" earthquake monitoring networks such as California Integrated Seismic Network serve a dual purpose.
First and foremost, they are the nations early warning system for disaster mitigation.
Secondly they are a research and analysis tool used by geophysicist to study earthquake propagation, localisation and clasification.
Application of Napali to earthquake detection could:
\begin{enumerate}
    \item Reduce the bandwidth requirement for operating a seismic sensor.
    \item Preserve subthrehold earthquake data for earthquake analysis.
    \item Reject single point anomalies resulting from human activity.
\end{enumerate}
Napali approach would aggregate multiple measurements, in order to conserve bandwidth for below threshold events.
If a local threshold is passed the measurement would be forwarded to the sink immediately in order to provide a timely latency.
If the sink determines that a P-wave has been detected, raw, high resolution data can be requested from the seismic sensor.

Currently IOT approaches are useless for scientific application, since only a "prewarning" is transmitted to the sink without the high resolution waveform.
With Napali useful high resolution data can be transmitted for later scientific analysis.