\begin{abstract}
Today's big data world is heavily relied on to bring precise, timely, and actionable intelligence, while being burdened by the ever increasing need for data cleaning and preprocessing.
While in the case of ingesting large quantity of unstructured data this problem is unavoidable, when it comes to sensor networks built for a specific purpose, such as anomaly detection, some of that computation can be moved to the edge of the network.
This thesis concerns the special case of sensor networks tailored for monitoring the power grid for anomalous behavior.
These networks consist of meters connected to the grid across multiple geographically separated locations, while monitoring the power delivery infrastructure with the intent of finding deviations from the nominal steady state.
Aforementioned deviations, known as power quality anomalies, may originate, and be localized to the location of the sensor, or may affect a sizable portion of the power grid.
The difficulty of evaluating the extent of a power quality anomaly stems directly from their short temporal and variable geographical impact.
I present a novel distributed power quality monitoring system called Napali which relies on extracted metrics from individual meters and their temporal locality in order to intelligently detect anomalies and extract raw data within temporal window and geographical areas of interest.
The results of this research are useful in other disciplines, such as general sensor network applications, IOT, and intrusion detection systems.
\end{abstract}
